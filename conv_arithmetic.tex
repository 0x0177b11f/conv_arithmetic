\documentclass{article}
\usepackage{amsmath,amsfonts}
\usepackage{authblk}
\usepackage[T1]{fontenc}
\usepackage[letterpaper, margin=1.5in]{geometry}
\usepackage{graphicx}
\usepackage{hyperref}
\usepackage[utf8]{inputenc}
\usepackage{natbib}
\usepackage{xcolor}

\definecolor{red}{RGB}{220,50,47}

\newcommand{\todo}[1]{\textcolor{red}{TODO: #1}}

\title{A guide to convolution arithmetic for deep learning}
\author[1]{Vincent Dumoulin\thanks{dumouliv@iro.umontreal.ca}}
\author[1]{Francesco Visin\thanks{fvisin@gmail.com}}
\affil[1]{MILA, Universit\'{e} de Montr\'{e}al}
\date{\today}

\begin{document}

\maketitle

\begin{abstract}
\todo{WRITEME}
\end{abstract}

\section{Introduction}

Deep convolutional neural networks (CNNs) have been at the heard of spectacular
advances in deep learning. Although CNNs have been used as early as the nineties
\citep{lecun1998gradient} to solve character recognition tasks, their current
widespread application is due to much more recent work, when a deep CNN was used
to beat state-of-the-art in the ImageNet image classification challenge
\citep{krizhevsky2012imagenet}.

Convolutional neural networks therefore constitute a very useful tool for
machine learning practitioners. However, learning to use CNNs for the first time
is generally an intimidating experience. A convolutional layer's output shape is
affected by the shape of its input as well as the choice of kernel shape, zero
padding and strides, and the relationship between these properties is not
trivial to infer. This contrasts with fully-connected layers, whose output size
is independent of its input size.

Additionally, so-called transposed convolutional layers (also known as
fractionally strided convolutional layers) have been employed in more and more
work as of late, and their relationship with convolutional layers has been
explained with various degrees of clarity.

This guide's objective is twofold:

\begin{enumerate}
    \item Explain the relationship between convolutional layers and transposed
        convolutional layers.
    \item Provide an intuitive understanding of the relationship between input
        shape, kernel shape, zero padding, strides and output shape in
        convolutional layers and transposed convolutional layers.
\end{enumerate}

\bibliography{bibliography}
\bibliographystyle{natbib}
\end{document}
